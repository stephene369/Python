\newpage
\subsubsection{Outils d'analse des données}
\begin{itemize}
	\item[$\bullet$] \textbf{Les tendances}
    \par{La tendance est considérée comme étant la pierre angulaire de l'analyse technique par les traders, et 
    dénote la direction d'un marché à un moment donné indiquant la tendance de la variation des prix.
    On distingue trois catégories de tendances à savoir, la tendance haussière, la tendance baissière et 
    la tendance neutre. 

    Une tendance haussière en bourse est une direction générale à la hausse des prix d'un actif sur une période
    donné. Elle se produit lorsque les prix d'un actif augmente régulièrement sur une période de temps donnée.
    Certains investisseur optimistes achètent souvent dans une tendance haussière dans l'espoir de réaliser des 
    gains à mesure que le prix continue d'augmenter.
    Une tendance à la baisse est une situation boursière dans laquelle le prix d'un actif financier baisse pendant 
    une certaine période[5]. 
    Une tendance latéral se produit lorsque le prix d'un actif évolue horizontalement, sans tendance nette à la 
    hausse ou à la baisse.
    Une tendance en dent de scie se caractérise se caractérise par des creux et des sommets qui se forment à des 
    niveaux différents sans direction claire.
    Une tendance est un inversion de l'évolution des cours d'un actif : lorsqu'une tendance haussière se transforme
    en tendance baissière et vise versa. }

    \item[$\bullet$] \textbf{Les niveaux de support et de résistance.}
    \par{Les niveaux de résistance et de support illustre la manière dont les forces de l'offre et de la demande interagissent 
    pour déterminer le prix en vigueur d'un actifs sous-jacent[4]. La résistance est une ligne qui relie les sommets des 
    points les plus hauts de la courbe des actifs. Le support est une ligne qui relie les sommets des 
    points les plus bas de la courbe. Si une ligne touche au moins trois points, elle devient significative.
    Plus les cours viendront toucher les lignes de résistance et de support, plus elles résisteront aux hausses et aux 
    chutes du cours des actions. Les Lignes de résistances et de support peuvent être utilisées pour délimiter les zones 
    potentielles où les prix peuvent rencontrer des changements de directions.

    Lorsque les cours arrivent finalement à traverser la résistance cela voudrait dire qu'ils ont accumulé
    assez de force pour traverser la résistance et qu'un changement significatif existe.
    Si les cours traversent le support cela implique une diminution potentiellement significative du cours de l'actifs.}

	
	\item[$\bullet$] \textbf{Les Moyennes Mobiles.}
	\par{La moyenne mobile est une ligne de tendance qui donne une idée de l'évolution des cours du marché.
	À chaque point de la moyenne mobile, la valeur est un indicateur du prix moyen sur une période donnée.
	Parfois il s'agit de la moyenne mobile arithmétique, d'autres fois ce sont des Moyennes Mobiles plus complexes qui sont 
	utilisées, comme la moyenne mobile exponentielle. La configuration et la mise en œuvre de cet indicateur 
	nécessitent un paramètre très important, qui est la période de temps.
	En trading, les Moyennes Mobiles sont utilisées pour générer des signaux d'achat et des signaux de vente. 
	Dans la plupart des cas, deux Moyennes Mobiles sont utilisées dans le but de générer les signaux : 
	la moyenne mobile rapide et la moyenne lente.}

	\par{La moyenne mobile rapide, comparativement à la moyenne mobile lente, est calculée sur une période plus courte. 
	Vient donc une question primordiale qui est comment choisir la période des deux Moyennes Mobiles 
	qui répond au mieux à nos données? Pour la moyenne mobile rapide, on peut choisir une moyenne mobile qui 
	sert de support à la correction d'une remontée franche arrivant après un plus bas, et pour la 
	moyenne mobile lente, le double de la période de la moyenne mobile rapide. [1] Étant donné que chaque valeur sur le marché 
	peut avoir ses propres caractéristiques uniques, indépendamment des tendances générales du marché, il est nécessaire
	d'utiliser plusieurs périodes de Moyennes Mobiles pour trouver la période qui convient le mieux à l'analyse 
	du titre spécifié. Par ailleurs, il faut garder à l'esprit de toujours garder une différence considérable entre 
	les périodes des Moyennes Mobiles rapides et les périodes des moyennes lentes, pour éviter que les courbes ne se 
	chevauchent et donc donnent des signaux contradictoires. En effet, suivant les valeurs de la différence entre
	la période lente et la période rapide, plus la différence est grande, plus les Moyennes Mobiles s'éloigneront les 
	unes des autres. Cette différence est l'un des paramètres permettant de définir le profil d'investissement (spéculation,
	investissement à moyen terme, investissement à long terme). }


    \item[$\bullet$] \textbf{Stratégie des moyennes mobile.}
    \par{La stratégie des Moyennes Mobiles utilise les Moyennes Mobiles arithmétiques 
    sur deux différentes périodes : la moyenne mobile lente (moyenne longue) et la moyenne mobile rapide (courte).
    En effet, la moyenne mobile arithmétique est un indicateur technique utilisé pour lisser les données de prix sur une
    période donnée. Son objectif principal est de fournir une estimation de la tendance générale des prix en filtrant les fluctuations.
    Suivant les différents croisements des deux Moyennes Mobiles, on peut détecter des signaux d'achat et de vente.
    Ainsi, lorsque la moyenne mobile rapide croise et dépasse la moyenne mobile lente, alors le marché est en tendance haussière 
    ce qui est un signal d'achat. Par contre, quand la moyenne mobile rapide croise et est inférieure à la moyenne mobile lente,
    alors le marché est en tendance baissière et cela implique un signal de vente. 
    Si les deux moyennes sont confondues, nous sommes en range.
    La création de cette stratégie est généralement attribuée à \textbf{Charles H. Dow}, un journaliste financier américain et 
    fondateur du Dow Jones \& Company. Il est l'un des pionniers de l'analyse technique des marchés financiers et de 
    l'utilisation des Moyennes Mobiles.}

    \item[$\bullet$] \textbf{Strategie de la Moyenne mobile convergence divergence.}
    \par{L'indicateur MACD (Moving Average Convergence Divergence) est un indicateur de suivi de tendance 
    utilisé pour localiser les tendances du marché. Il est composé d'un histogramme de deux lignes 
    calculées à partir des Moyennes Mobiles exponentielles et d'une ligne MACD et d'une ligne signal.
    Les deux lignes de Moyennes Mobiles Exponentielles sont calculées sur les cours de clôture. 
    En effet, la Moyenne Mobile Exponentielle est un indicateur technique qui permet de lisser les cours
    des actifs en affectant plus de poids aux prix les plus récents tout en diminuant progressivement le
    poids des prix plus anciens.
    La ligne MACD est le résultat de la différence entre les Moyennes Mobiles rapides qui sont de courtes périodes et la moyenne 
    mobile lente qui est de longue période. 
    Une MACD positive indique une tendance haussière et une MACD négative indique une tendance baissière.
    La ligne signal est la moyenne mobile arithmétique de la ligne MACD. Cette ligne lisse la MACD et permet
    de générer les signaux d'achat et de vente.
    Ainsi, suivant les valeurs de la ligne signal et de la ligne MACD, lorsque la ligne MACD croise
    la ligne de signal et la dépasse, cela génère un signal d'achat et dans le cas où la ligne MACD croise 
    la ligne signal et passe en-dessous de celle-ci, cela génère un signal de vente.
    La MACD a été développée par \textbf{Gérald Appel}, un analyste financier et trader américain connu pour ses
    contributions à l'analyse technique. Il faut noter que l'indicateur est le résultat de nombreuses
    contributions et améliorations apportées par divers professionnels de l'analyse technique au fil des ans.
    Cependant, étant donné que la Moyenne Mobile est un type d'indicateur technique qui se base sur des 
    données historiques pour générer des signaux (indicateur retard), il est donc par conséquent lent à réagir aux
    changements de prix comparativement aux indicateurs avancés qui anticipent les mouvements futurs des prix. 
    Pour cela, il est important de combiner la MACD avec d'autres indicateurs afin de confirmer les signaux.}
    \newpage
    \item[$\bullet$] \textbf{Strategie de l' oscillateur stochastique.}
    \par{L'oscillateur stochastique est un indicateur qui évalue les conditions de surachat et de survente
    du marché. Le stochastique est basé sur le fait que lorsque les prix suivent une tendance haussière, les prix 
    de fermeture tendent à se rapprocher du niveau supérieur de l'écart des prix. de même lorsque 
    la tendance des prix est en baisse, les prix de fermeture tendent à se rapprocher du 
    niveau inférieur de l'écart des prix.
    Pour cela, il fait appel à deux lingnes nommées \%K et \%D. 
    Le \%D est le plus plus important dans le sens ou c'est lui qui donne le signal.
    Ces lignes permettent de savoir si le marché est dans un état de surachat ou dans un état de survente[3].
    En effet, la ligne \%K est la ligne principale de l'oscillateur stochastique. Elle représente la position
    actuelle du prix de clôture par rapport à un échantillon de prix sur une période donnée.
    La ligne \%D, quant à elle, est la moyenne mobile arithmétique de la ligne \%K sur une autre période.
    Cette ligne est lissée afin de donner une meilleure indication sur les conditions du marché.
    Suivant les deux lignes, une valeur de \%K et de \%D de moins de 30\% indique que nous sommes dans un état de survente, et une valeur
    de plus de 70\% indique un marché dans un état de surachat. 
    Développé par \textbf{George C. Lane} vers les années 1950, l'oscillateur stochastique est utilisé pour identifier
    les changements potentiels dans les tendances et pour confirmer les signaux générés par d'autres indicateurs.}
    % paragraph Ceci est  (end)

\end{itemize}

\subsubsection{Critère de decision }
\begin{itemize}
\item[$\bullet$] \textbf{Le Backtesting.}
\par{Le Backtesting est une façon d'analyser la performance potentielle d'une stratégie de 
trading en l'appliquant à des données historiques réelles. C'est un outil qui aide à choisir
la stratégie présentant le meilleur résultat. Cette stratégie se base sur l'idée selon laquelle
une stratégie ayant donné de meilleurs résultats sur des données passées est susceptible de le faire également
dans les conditions actuelles ou futures du marché. En effet, le Backtesting permet de tester les stratégies
tout en ajustant les paramètres afin d'obtenir les meilleurs résultats. Les tests sont réalisés très rapidement
et sans risquer le capital.}
\end{itemize}
%Pour réaliser la backtesting, il faut definir la stratégie de trading , collecter les données historique

\subsubsection{Limite de l'étude. }
\par{Notre étude compare uniquement les performances de deux stratégies de trading sur les indices BRVM-Agriculture et 
BRVM-Service Publique. Malheureusement, les résultats de notre étude ne peuvent pas être utilisés pour d'autres 
indices ou pour d'autres titres sur d'autres marchés, y compris même celui de la BRVM. De plus, les périodes utilisées
pour les calculs de moyenne mobile sont de plus de 20 jours, ce qui n'est pas convenable pour les traders spéculateurs.
Il faut noter que le Backtesting, bien qu'il permette de trouver la stratégie la plus performante, ne garantit cependant pas
la fiabilité de celle-ci sur les données futures du marché.}

