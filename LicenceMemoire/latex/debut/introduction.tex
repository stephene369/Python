\chapter*{Introduction}
\addcontentsline{toc}{chapter}{\large Introduction}

\par{
    Dans un monde de plus en plus orienté vers l'économie de la 
    financiarisation, on observe une croissance notable du nombre 
    d'actifs financiers ainsi que de celui des investisseurs.
    Généralement caractérisée par une série de tendances et de 
    transformations économiques et financières, elle contribue 
    surtout à l'accès à un financement supplémentaire, 
    une grande visibilité et une crédibilité accrue pour les entreprises 
    \cite{1}.
    Dans le même temps, elle favorise globalement l'accroissement du 
    revenu des investisseurs, la préservation du capital et une 
    diversification des investissements aux investisseurs.
    La BRVM (Bourse Régionale des Valeurs Mobilières) est un marché 
    financier régional pour les pays de l'UEMOA. En tant que marché 
    financier, il joue un rôle essentiel dans le financement des 
    entreprises des États de l'Afrique de l'Ouest. En 2023 la 
    capitalisation boursière des actions, droits et obligations de 
    la BRVM s'élèvent à plus de 16.5 billions de FCFA 
    \cite{3}.
    Grâce aux avancées technologiques, il est aujourd'hui possible de 
    négocier des actions et d'autres titres financiers via des plateformes 
    de trading électroniques sur Internet. Cette nouvelle façon d'acheter 
    et de revendre des actifs financiers a donnée naissance au trading 
    algorithmique. Plusieurs méthodes et outils d'analyse technique existent
    et son disponible sur certaines plateformes de trading afin de 
    permettre aux investisseurs de prendre des décisions. En effet, il 
    existe autant de stratégies de trading dans le monde qu'en existe de 
    traders. Mais ils peuvent être regroupés en dix (10) catégories, pour 
    un chiffre avoisinant cent (100) comme nombre total d'outils et 
    méthode d'analyse technique officiel et répertoriés dans les ouvrages 
    \cite{2}. }

\par{
    Cependant bien que ces outils d'analyse technique sont très utiles dans 
    la prise de décisions en trading, il est important de se rappeler qu'ils
     ne sont pas infaillibles et que les marchés financiers peuvent parfois 
    réagir de manière imprévisible. Au vu du grand nombre de ces outils 
    d'analyse technique, il est souvent très difficile aux investisseurs de 
    savoir lequelle utilisé.
    }

\par{
    Toutefois certains outils permettent me mesurer les performances d'une 
    méthode d'analyse technique sur une valeur précise. 
    Il s'agira dans cette étude de comparer la stratégie de trading basé 
    sur les moyennes mobiles à la stratégie combinée de l'oscillateur 
    stochastique et de la moyenne mobile convergence divergence, sur les 
    indices BRVM-Agriculture et BRVM-Services-Publics afin d'aider les 
    investisseurs de la BRVM qui investissent dans les secteurs de 
    l'Agriculute et des Services Publics à la BRVM de pourvoir choisir 
    facilement une stratégie. Aussi, un outil de comparaison de ces deux 
    stratégies sera mis en place afin de permettre aux investisseurs de 
    pouvoir comparer les deux stratégies sur tous les actifs financier.
    Dans le premier chapitre intitulé cadre institutionnel et observation 
    du stage, nous allons présenter le cadre institutionnel qu'est LESCAL 
    puis dans le chapitre 2, nous allons parler de la méthodologie puis 
    enfin dans le chapitre trois, nous allons appliquer les stratégies de 
    trading sur les différentes données et présenter les résultats et 
    vérification des hypothèses.
}
